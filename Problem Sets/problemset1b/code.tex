% Options for packages loaded elsewhere
\PassOptionsToPackage{unicode}{hyperref}
\PassOptionsToPackage{hyphens}{url}
%
\documentclass[
]{article}
\usepackage{amsmath,amssymb}
\usepackage{iftex}
\ifPDFTeX
  \usepackage[T1]{fontenc}
  \usepackage[utf8]{inputenc}
  \usepackage{textcomp} % provide euro and other symbols
\else % if luatex or xetex
  \usepackage{unicode-math} % this also loads fontspec
  \defaultfontfeatures{Scale=MatchLowercase}
  \defaultfontfeatures[\rmfamily]{Ligatures=TeX,Scale=1}
\fi
\usepackage{lmodern}
\ifPDFTeX\else
  % xetex/luatex font selection
\fi
% Use upquote if available, for straight quotes in verbatim environments
\IfFileExists{upquote.sty}{\usepackage{upquote}}{}
\IfFileExists{microtype.sty}{% use microtype if available
  \usepackage[]{microtype}
  \UseMicrotypeSet[protrusion]{basicmath} % disable protrusion for tt fonts
}{}
\makeatletter
\@ifundefined{KOMAClassName}{% if non-KOMA class
  \IfFileExists{parskip.sty}{%
    \usepackage{parskip}
  }{% else
    \setlength{\parindent}{0pt}
    \setlength{\parskip}{6pt plus 2pt minus 1pt}}
}{% if KOMA class
  \KOMAoptions{parskip=half}}
\makeatother
\usepackage{xcolor}
\usepackage[margin=1in]{geometry}
\usepackage{color}
\usepackage{fancyvrb}
\newcommand{\VerbBar}{|}
\newcommand{\VERB}{\Verb[commandchars=\\\{\}]}
\DefineVerbatimEnvironment{Highlighting}{Verbatim}{commandchars=\\\{\}}
% Add ',fontsize=\small' for more characters per line
\usepackage{framed}
\definecolor{shadecolor}{RGB}{248,248,248}
\newenvironment{Shaded}{\begin{snugshade}}{\end{snugshade}}
\newcommand{\AlertTok}[1]{\textcolor[rgb]{0.94,0.16,0.16}{#1}}
\newcommand{\AnnotationTok}[1]{\textcolor[rgb]{0.56,0.35,0.01}{\textbf{\textit{#1}}}}
\newcommand{\AttributeTok}[1]{\textcolor[rgb]{0.13,0.29,0.53}{#1}}
\newcommand{\BaseNTok}[1]{\textcolor[rgb]{0.00,0.00,0.81}{#1}}
\newcommand{\BuiltInTok}[1]{#1}
\newcommand{\CharTok}[1]{\textcolor[rgb]{0.31,0.60,0.02}{#1}}
\newcommand{\CommentTok}[1]{\textcolor[rgb]{0.56,0.35,0.01}{\textit{#1}}}
\newcommand{\CommentVarTok}[1]{\textcolor[rgb]{0.56,0.35,0.01}{\textbf{\textit{#1}}}}
\newcommand{\ConstantTok}[1]{\textcolor[rgb]{0.56,0.35,0.01}{#1}}
\newcommand{\ControlFlowTok}[1]{\textcolor[rgb]{0.13,0.29,0.53}{\textbf{#1}}}
\newcommand{\DataTypeTok}[1]{\textcolor[rgb]{0.13,0.29,0.53}{#1}}
\newcommand{\DecValTok}[1]{\textcolor[rgb]{0.00,0.00,0.81}{#1}}
\newcommand{\DocumentationTok}[1]{\textcolor[rgb]{0.56,0.35,0.01}{\textbf{\textit{#1}}}}
\newcommand{\ErrorTok}[1]{\textcolor[rgb]{0.64,0.00,0.00}{\textbf{#1}}}
\newcommand{\ExtensionTok}[1]{#1}
\newcommand{\FloatTok}[1]{\textcolor[rgb]{0.00,0.00,0.81}{#1}}
\newcommand{\FunctionTok}[1]{\textcolor[rgb]{0.13,0.29,0.53}{\textbf{#1}}}
\newcommand{\ImportTok}[1]{#1}
\newcommand{\InformationTok}[1]{\textcolor[rgb]{0.56,0.35,0.01}{\textbf{\textit{#1}}}}
\newcommand{\KeywordTok}[1]{\textcolor[rgb]{0.13,0.29,0.53}{\textbf{#1}}}
\newcommand{\NormalTok}[1]{#1}
\newcommand{\OperatorTok}[1]{\textcolor[rgb]{0.81,0.36,0.00}{\textbf{#1}}}
\newcommand{\OtherTok}[1]{\textcolor[rgb]{0.56,0.35,0.01}{#1}}
\newcommand{\PreprocessorTok}[1]{\textcolor[rgb]{0.56,0.35,0.01}{\textit{#1}}}
\newcommand{\RegionMarkerTok}[1]{#1}
\newcommand{\SpecialCharTok}[1]{\textcolor[rgb]{0.81,0.36,0.00}{\textbf{#1}}}
\newcommand{\SpecialStringTok}[1]{\textcolor[rgb]{0.31,0.60,0.02}{#1}}
\newcommand{\StringTok}[1]{\textcolor[rgb]{0.31,0.60,0.02}{#1}}
\newcommand{\VariableTok}[1]{\textcolor[rgb]{0.00,0.00,0.00}{#1}}
\newcommand{\VerbatimStringTok}[1]{\textcolor[rgb]{0.31,0.60,0.02}{#1}}
\newcommand{\WarningTok}[1]{\textcolor[rgb]{0.56,0.35,0.01}{\textbf{\textit{#1}}}}
\usepackage{graphicx}
\makeatletter
\def\maxwidth{\ifdim\Gin@nat@width>\linewidth\linewidth\else\Gin@nat@width\fi}
\def\maxheight{\ifdim\Gin@nat@height>\textheight\textheight\else\Gin@nat@height\fi}
\makeatother
% Scale images if necessary, so that they will not overflow the page
% margins by default, and it is still possible to overwrite the defaults
% using explicit options in \includegraphics[width, height, ...]{}
\setkeys{Gin}{width=\maxwidth,height=\maxheight,keepaspectratio}
% Set default figure placement to htbp
\makeatletter
\def\fps@figure{htbp}
\makeatother
\setlength{\emergencystretch}{3em} % prevent overfull lines
\providecommand{\tightlist}{%
  \setlength{\itemsep}{0pt}\setlength{\parskip}{0pt}}
\setcounter{secnumdepth}{-\maxdimen} % remove section numbering
\ifLuaTeX
  \usepackage{selnolig}  % disable illegal ligatures
\fi
\IfFileExists{bookmark.sty}{\usepackage{bookmark}}{\usepackage{hyperref}}
\IfFileExists{xurl.sty}{\usepackage{xurl}}{} % add URL line breaks if available
\urlstyle{same}
\hypersetup{
  pdftitle={code.R},
  pdfauthor={anirudh},
  hidelinks,
  pdfcreator={LaTeX via pandoc}}

\title{code.R}
\author{anirudh}
\date{2023-09-12}

\begin{document}
\maketitle

\begin{Shaded}
\begin{Highlighting}[]
\CommentTok{\#library(car)}

\DocumentationTok{\#\#\#\#\#\#\#\#\#\#\#\#\#\#\#\#\#\#\#\#\#\#\#\#\#\#\#\#\#\#\#\#\#}

\CommentTok{\# Part 1: R Questions}

\DocumentationTok{\#\#\#\#\#\#\#\#\#\#\#\#\#\#\#\#\#\#\#\#\#\#\#\#\#\#\#\#\#\#\#\#\#}

\CommentTok{\# Question 1: Loading, summarizing and plotting the dataset}

\NormalTok{dataframe }\OtherTok{\textless{}{-}} \FunctionTok{read.csv}\NormalTok{(}\StringTok{"Advertising.csv"}\NormalTok{)}
\FunctionTok{summary}\NormalTok{(dataframe)}
\end{Highlighting}
\end{Shaded}

\begin{verbatim}
##        X                TV             Radio          Newspaper     
##  Min.   :  1.00   Min.   :  0.70   Min.   : 0.000   Min.   :  0.30  
##  1st Qu.: 50.75   1st Qu.: 74.38   1st Qu.: 9.975   1st Qu.: 12.75  
##  Median :100.50   Median :149.75   Median :22.900   Median : 25.75  
##  Mean   :100.50   Mean   :147.04   Mean   :23.264   Mean   : 30.55  
##  3rd Qu.:150.25   3rd Qu.:218.82   3rd Qu.:36.525   3rd Qu.: 45.10  
##  Max.   :200.00   Max.   :296.40   Max.   :49.600   Max.   :114.00  
##      Sales      
##  Min.   : 1.60  
##  1st Qu.:10.38  
##  Median :12.90  
##  Mean   :14.02  
##  3rd Qu.:17.40  
##  Max.   :27.00
\end{verbatim}

\begin{Shaded}
\begin{Highlighting}[]
\FunctionTok{plot}\NormalTok{(dataframe)}
\end{Highlighting}
\end{Shaded}

\includegraphics{code_files/figure-latex/unnamed-chunk-1-1.pdf}

\begin{Shaded}
\begin{Highlighting}[]
\DocumentationTok{\#\#\#\#\#\#\#\#\#\#\#\#\#\#\#\#\#\#\#\#\#\#\#\#\#\#\#\#\#\#\#\#\#}

\CommentTok{\# Question 2: Simple Linear Regression}

\CommentTok{\# Yes, there is a relationship between sales and the mediums of advertisement. }
\CommentTok{\# TV and Sales have a clear linear relationship. With more advertisements on TV, the sales are almost proportionally high.}
\CommentTok{\# Radio and Sales also share somewhat of a relationship, however, it isn\textquotesingle{}t as linear as with TV and Sales. }
\CommentTok{\# Newspaper and Sales don\textquotesingle{}t show much of a relationship. Which means investing much of the advertising budget in Newspapers will not be worthwhile.}

\FunctionTok{print}\NormalTok{(dataframe[}\DecValTok{5}\NormalTok{])}
\end{Highlighting}
\end{Shaded}

\begin{verbatim}
##     Sales
## 1    22.1
## 2    10.4
## 3     9.3
## 4    18.5
## 5    12.9
## 6     7.2
## 7    11.8
## 8    13.2
## 9     4.8
## 10   10.6
## 11    8.6
## 12   17.4
## 13    9.2
## 14    9.7
## 15   19.0
## 16   22.4
## 17   12.5
## 18   24.4
## 19   11.3
## 20   14.6
## 21   18.0
## 22   12.5
## 23    5.6
## 24   15.5
## 25    9.7
## 26   12.0
## 27   15.0
## 28   15.9
## 29   18.9
## 30   10.5
## 31   21.4
## 32   11.9
## 33    9.6
## 34   17.4
## 35    9.5
## 36   12.8
## 37   25.4
## 38   14.7
## 39   10.1
## 40   21.5
## 41   16.6
## 42   17.1
## 43   20.7
## 44   12.9
## 45    8.5
## 46   14.9
## 47   10.6
## 48   23.2
## 49   14.8
## 50    9.7
## 51   11.4
## 52   10.7
## 53   22.6
## 54   21.2
## 55   20.2
## 56   23.7
## 57    5.5
## 58   13.2
## 59   23.8
## 60   18.4
## 61    8.1
## 62   24.2
## 63   15.7
## 64   14.0
## 65   18.0
## 66    9.3
## 67    9.5
## 68   13.4
## 69   18.9
## 70   22.3
## 71   18.3
## 72   12.4
## 73    8.8
## 74   11.0
## 75   17.0
## 76    8.7
## 77    6.9
## 78   14.2
## 79    5.3
## 80   11.0
## 81   11.8
## 82   12.3
## 83   11.3
## 84   13.6
## 85   21.7
## 86   15.2
## 87   12.0
## 88   16.0
## 89   12.9
## 90   16.7
## 91   11.2
## 92    7.3
## 93   19.4
## 94   22.2
## 95   11.5
## 96   16.9
## 97   11.7
## 98   15.5
## 99   25.4
## 100  17.2
## 101  11.7
## 102  23.8
## 103  14.8
## 104  14.7
## 105  20.7
## 106  19.2
## 107   7.2
## 108   8.7
## 109   5.3
## 110  19.8
## 111  13.4
## 112  21.8
## 113  14.1
## 114  15.9
## 115  14.6
## 116  12.6
## 117  12.2
## 118   9.4
## 119  15.9
## 120   6.6
## 121  15.5
## 122   7.0
## 123  11.6
## 124  15.2
## 125  19.7
## 126  10.6
## 127   6.6
## 128   8.8
## 129  24.7
## 130   9.7
## 131   1.6
## 132  12.7
## 133   5.7
## 134  19.6
## 135  10.8
## 136  11.6
## 137   9.5
## 138  20.8
## 139   9.6
## 140  20.7
## 141  10.9
## 142  19.2
## 143  20.1
## 144  10.4
## 145  11.4
## 146  10.3
## 147  13.2
## 148  25.4
## 149  10.9
## 150  10.1
## 151  16.1
## 152  11.6
## 153  16.6
## 154  19.0
## 155  15.6
## 156   3.2
## 157  15.3
## 158  10.1
## 159   7.3
## 160  12.9
## 161  14.4
## 162  13.3
## 163  14.9
## 164  18.0
## 165  11.9
## 166  11.9
## 167   8.0
## 168  12.2
## 169  17.1
## 170  15.0
## 171   8.4
## 172  14.5
## 173   7.6
## 174  11.7
## 175  11.5
## 176  27.0
## 177  20.2
## 178  11.7
## 179  11.8
## 180  12.6
## 181  10.5
## 182  12.2
## 183   8.7
## 184  26.2
## 185  17.6
## 186  22.6
## 187  10.3
## 188  17.3
## 189  15.9
## 190   6.7
## 191  10.8
## 192   9.9
## 193   5.9
## 194  19.6
## 195  17.3
## 196   7.6
## 197   9.7
## 198  12.8
## 199  25.5
## 200  13.4
\end{verbatim}

\begin{Shaded}
\begin{Highlighting}[]
\CommentTok{\# Running a simple regression over each of the variables}
\NormalTok{lm\_model\_TV }\OtherTok{\textless{}{-}} \FunctionTok{lm}\NormalTok{(}\FunctionTok{unlist}\NormalTok{(dataframe[}\DecValTok{5}\NormalTok{]) }\SpecialCharTok{\textasciitilde{}} \FunctionTok{unlist}\NormalTok{(dataframe[}\DecValTok{2}\NormalTok{]), }\AttributeTok{data =}\NormalTok{ dataframe)}
\FunctionTok{summary}\NormalTok{(lm\_model\_TV)}
\end{Highlighting}
\end{Shaded}

\begin{verbatim}
## 
## Call:
## lm(formula = unlist(dataframe[5]) ~ unlist(dataframe[2]), data = dataframe)
## 
## Residuals:
##     Min      1Q  Median      3Q     Max 
## -8.3860 -1.9545 -0.1913  2.0671  7.2124 
## 
## Coefficients:
##                      Estimate Std. Error t value Pr(>|t|)    
## (Intercept)          7.032594   0.457843   15.36   <2e-16 ***
## unlist(dataframe[2]) 0.047537   0.002691   17.67   <2e-16 ***
## ---
## Signif. codes:  0 '***' 0.001 '**' 0.01 '*' 0.05 '.' 0.1 ' ' 1
## 
## Residual standard error: 3.259 on 198 degrees of freedom
## Multiple R-squared:  0.6119, Adjusted R-squared:  0.6099 
## F-statistic: 312.1 on 1 and 198 DF,  p-value: < 2.2e-16
\end{verbatim}

\begin{Shaded}
\begin{Highlighting}[]
\FunctionTok{plot}\NormalTok{(}\FunctionTok{unlist}\NormalTok{(dataframe[}\DecValTok{2}\NormalTok{]), }\FunctionTok{unlist}\NormalTok{(dataframe[}\DecValTok{5}\NormalTok{]))}
\FunctionTok{abline}\NormalTok{(lm\_model\_TV, }\AttributeTok{col =} \StringTok{"red"}\NormalTok{)}
\end{Highlighting}
\end{Shaded}

\includegraphics{code_files/figure-latex/unnamed-chunk-1-2.pdf}

\begin{Shaded}
\begin{Highlighting}[]
\NormalTok{lm\_model\_RADIO }\OtherTok{\textless{}{-}} \FunctionTok{lm}\NormalTok{(}\FunctionTok{unlist}\NormalTok{(dataframe[}\DecValTok{5}\NormalTok{]) }\SpecialCharTok{\textasciitilde{}} \FunctionTok{unlist}\NormalTok{(dataframe[}\DecValTok{3}\NormalTok{]), }\AttributeTok{data =}\NormalTok{ dataframe)}
\FunctionTok{summary}\NormalTok{(lm\_model\_RADIO)}
\end{Highlighting}
\end{Shaded}

\begin{verbatim}
## 
## Call:
## lm(formula = unlist(dataframe[5]) ~ unlist(dataframe[3]), data = dataframe)
## 
## Residuals:
##      Min       1Q   Median       3Q      Max 
## -15.7305  -2.1324   0.7707   2.7775   8.1810 
## 
## Coefficients:
##                      Estimate Std. Error t value Pr(>|t|)    
## (Intercept)           9.31164    0.56290  16.542   <2e-16 ***
## unlist(dataframe[3])  0.20250    0.02041   9.921   <2e-16 ***
## ---
## Signif. codes:  0 '***' 0.001 '**' 0.01 '*' 0.05 '.' 0.1 ' ' 1
## 
## Residual standard error: 4.275 on 198 degrees of freedom
## Multiple R-squared:  0.332,  Adjusted R-squared:  0.3287 
## F-statistic: 98.42 on 1 and 198 DF,  p-value: < 2.2e-16
\end{verbatim}

\begin{Shaded}
\begin{Highlighting}[]
\FunctionTok{plot}\NormalTok{(}\FunctionTok{unlist}\NormalTok{(dataframe[}\DecValTok{3}\NormalTok{]), }\FunctionTok{unlist}\NormalTok{(dataframe[}\DecValTok{5}\NormalTok{]))}
\FunctionTok{abline}\NormalTok{(lm\_model\_RADIO, }\AttributeTok{col =} \StringTok{"blue"}\NormalTok{)}
\end{Highlighting}
\end{Shaded}

\includegraphics{code_files/figure-latex/unnamed-chunk-1-3.pdf}

\begin{Shaded}
\begin{Highlighting}[]
\NormalTok{lm\_model\_NEWSPAPER }\OtherTok{\textless{}{-}} \FunctionTok{lm}\NormalTok{(}\FunctionTok{unlist}\NormalTok{(dataframe[}\DecValTok{5}\NormalTok{]) }\SpecialCharTok{\textasciitilde{}} \FunctionTok{unlist}\NormalTok{(dataframe[}\DecValTok{4}\NormalTok{]), }\AttributeTok{data =}\NormalTok{ dataframe)}
\FunctionTok{summary}\NormalTok{(lm\_model\_NEWSPAPER)}
\end{Highlighting}
\end{Shaded}

\begin{verbatim}
## 
## Call:
## lm(formula = unlist(dataframe[5]) ~ unlist(dataframe[4]), data = dataframe)
## 
## Residuals:
##      Min       1Q   Median       3Q      Max 
## -11.2272  -3.3873  -0.8392   3.5059  12.7751 
## 
## Coefficients:
##                      Estimate Std. Error t value Pr(>|t|)    
## (Intercept)          12.35141    0.62142   19.88  < 2e-16 ***
## unlist(dataframe[4])  0.05469    0.01658    3.30  0.00115 ** 
## ---
## Signif. codes:  0 '***' 0.001 '**' 0.01 '*' 0.05 '.' 0.1 ' ' 1
## 
## Residual standard error: 5.092 on 198 degrees of freedom
## Multiple R-squared:  0.05212,    Adjusted R-squared:  0.04733 
## F-statistic: 10.89 on 1 and 198 DF,  p-value: 0.001148
\end{verbatim}

\begin{Shaded}
\begin{Highlighting}[]
\FunctionTok{plot}\NormalTok{(}\FunctionTok{unlist}\NormalTok{(dataframe[}\DecValTok{4}\NormalTok{]), }\FunctionTok{unlist}\NormalTok{(dataframe[}\DecValTok{5}\NormalTok{]))}
\FunctionTok{abline}\NormalTok{(lm\_model\_NEWSPAPER, }\AttributeTok{col =} \StringTok{"green"}\NormalTok{)}
\end{Highlighting}
\end{Shaded}

\includegraphics{code_files/figure-latex/unnamed-chunk-1-4.pdf}

\begin{Shaded}
\begin{Highlighting}[]
\CommentTok{\# We see from the graphs that the coefficients of TV and Sales model have a good fit. Radio and Sales have an average fit.}
\CommentTok{\# And Newspaper and Sales has the worst fit. As for each medium\textquotesingle{}s contribution to sales, TV and Radio definitely contribute, but}
\CommentTok{\# Newspaper doesn\textquotesingle{}t seem to. }

\DocumentationTok{\#\#\#\#\#\#\#\#\#\#\#\#\#\#\#\#\#\#\#\#\#\#\#\#\#\#\#\#\#\#\#\#\#}

\CommentTok{\# Question 3: Multiple Linear Regression}

\NormalTok{mult\_lm\_model }\OtherTok{\textless{}{-}} \FunctionTok{lm}\NormalTok{(}\FunctionTok{unlist}\NormalTok{(dataframe[}\DecValTok{5}\NormalTok{]) }\SpecialCharTok{\textasciitilde{}} \FunctionTok{unlist}\NormalTok{(dataframe[}\DecValTok{4}\NormalTok{]) }\SpecialCharTok{+} \FunctionTok{unlist}\NormalTok{(dataframe[}\DecValTok{3}\NormalTok{]) }\SpecialCharTok{+} \FunctionTok{unlist}\NormalTok{(dataframe[}\DecValTok{2}\NormalTok{]), }\AttributeTok{data =}\NormalTok{ dataframe)}
\FunctionTok{summary}\NormalTok{(mult\_lm\_model)}
\end{Highlighting}
\end{Shaded}

\begin{verbatim}
## 
## Call:
## lm(formula = unlist(dataframe[5]) ~ unlist(dataframe[4]) + unlist(dataframe[3]) + 
##     unlist(dataframe[2]), data = dataframe)
## 
## Residuals:
##     Min      1Q  Median      3Q     Max 
## -8.8277 -0.8908  0.2418  1.1893  2.8292 
## 
## Coefficients:
##                       Estimate Std. Error t value Pr(>|t|)    
## (Intercept)           2.938889   0.311908   9.422   <2e-16 ***
## unlist(dataframe[4]) -0.001037   0.005871  -0.177     0.86    
## unlist(dataframe[3])  0.188530   0.008611  21.893   <2e-16 ***
## unlist(dataframe[2])  0.045765   0.001395  32.809   <2e-16 ***
## ---
## Signif. codes:  0 '***' 0.001 '**' 0.01 '*' 0.05 '.' 0.1 ' ' 1
## 
## Residual standard error: 1.686 on 196 degrees of freedom
## Multiple R-squared:  0.8972, Adjusted R-squared:  0.8956 
## F-statistic: 570.3 on 3 and 196 DF,  p-value: < 2.2e-16
\end{verbatim}

\begin{Shaded}
\begin{Highlighting}[]
\NormalTok{mult\_lm\_model}\SpecialCharTok{$}\NormalTok{coefficients}
\end{Highlighting}
\end{Shaded}

\begin{verbatim}
##          (Intercept) unlist(dataframe[4]) unlist(dataframe[3]) 
##          2.938889369         -0.001037493          0.188530017 
## unlist(dataframe[2]) 
##          0.045764645
\end{verbatim}

\begin{Shaded}
\begin{Highlighting}[]
\CommentTok{\# The coefficient of newspaper is negative while TV and Radio are positive. We also see the p{-}value given in the summary as}
\CommentTok{\# less than 2.2e{-}16 which means that coefficients are statistically significant because typically a p{-}value \textless{} 0.05 is considered}
\CommentTok{\# statistically significant. }

\CommentTok{\# Do they all contribute to sales?}
\CommentTok{\# Newspaper definitely doesn\textquotesingle{}t because of the negative relationship. But TV and Radio do due to the positive coefficients.}

\CommentTok{\# Reconciling results of multiple and simple regressions for newspaper}
\CommentTok{\# If we look at the coefficients of the simple Linear Regression\textquotesingle{}s model and compare it with the respective coefficients of the }
\CommentTok{\# Multiple Linear Regression models, they aren\textquotesingle{}t too far apart. It won\textquotesingle{}t be the exact same but will be close to each other }
\CommentTok{\# because in multiple Linear Regression model, it\textquotesingle{}s trying to fit it for all the three advertising mediums. }

\CommentTok{\# How strong is the relationship between advertising and sales?}
\CommentTok{\# It\textquotesingle{}s mostly okay because it\textquotesingle{}s not the strongest with Radio and Newspaper but if a business had to invest their budget }
\CommentTok{\# into advertisements for increasing their sales, then they should do it only in TV and Radio because they have good relationship}
\CommentTok{\# with sales. }

\CommentTok{\# Discussing R{-}squared results}
\CommentTok{\# The R{-}Squared value is computed to be 0.8972 or 89.72\% which is very good. It means that we got a good fit and the model is }
\CommentTok{\# able to accurately predict the output for 90\% of the data. However, it is also important to keep in mind to use other }
\CommentTok{\# metrics }

\CommentTok{\# Plotting a 3d graph of Sales, TV and Radio.}

\CommentTok{\#scatter3d(Sales\textasciitilde{}TV+Radio)}

\DocumentationTok{\#\#\#\#\#\#\#\#\#\#\#\#\#\#\#\#\#\#\#\#\#\#\#\#\#\#\#\#\#\#\#\#}
\CommentTok{\# Question 4: Models with interaction terms}

\NormalTok{lm\_model\_TV\_Radio }\OtherTok{\textless{}{-}} \FunctionTok{lm}\NormalTok{(}\FunctionTok{unlist}\NormalTok{(dataframe[}\DecValTok{5}\NormalTok{]) }\SpecialCharTok{\textasciitilde{}} \FunctionTok{unlist}\NormalTok{(dataframe[}\DecValTok{3}\NormalTok{]) }\SpecialCharTok{*} \FunctionTok{unlist}\NormalTok{(dataframe[}\DecValTok{2}\NormalTok{]), }\AttributeTok{data =}\NormalTok{ dataframe)}
\FunctionTok{summary}\NormalTok{(lm\_model\_TV\_Radio)}
\end{Highlighting}
\end{Shaded}

\begin{verbatim}
## 
## Call:
## lm(formula = unlist(dataframe[5]) ~ unlist(dataframe[3]) * unlist(dataframe[2]), 
##     data = dataframe)
## 
## Residuals:
##     Min      1Q  Median      3Q     Max 
## -6.3366 -0.4028  0.1831  0.5948  1.5246 
## 
## Coefficients:
##                                            Estimate Std. Error t value Pr(>|t|)
## (Intercept)                               6.750e+00  2.479e-01  27.233   <2e-16
## unlist(dataframe[3])                      2.886e-02  8.905e-03   3.241   0.0014
## unlist(dataframe[2])                      1.910e-02  1.504e-03  12.699   <2e-16
## unlist(dataframe[3]):unlist(dataframe[2]) 1.086e-03  5.242e-05  20.727   <2e-16
##                                              
## (Intercept)                               ***
## unlist(dataframe[3])                      ** 
## unlist(dataframe[2])                      ***
## unlist(dataframe[3]):unlist(dataframe[2]) ***
## ---
## Signif. codes:  0 '***' 0.001 '**' 0.01 '*' 0.05 '.' 0.1 ' ' 1
## 
## Residual standard error: 0.9435 on 196 degrees of freedom
## Multiple R-squared:  0.9678, Adjusted R-squared:  0.9673 
## F-statistic:  1963 on 3 and 196 DF,  p-value: < 2.2e-16
\end{verbatim}

\begin{Shaded}
\begin{Highlighting}[]
\CommentTok{\# R{-}squared = 0.9678 (or) 96.78; F{-}statistic = 1963}
\CommentTok{\# It seems like the R{-}Squared has gone up by a lot more. And the F{-}statistic is much higher which means it is }
\CommentTok{\# statistically significant and does a much better job of explaining the variation in the dependent variable, which means it estimates the output}
\CommentTok{\# quite precisely. So yes, there is a lot of synergy between TV and Radio due to the improved performance that we\textquotesingle{}ve observed.}

\CommentTok{\# Experimenting with variations in interaction terms}
\NormalTok{lm\_model\_TV\_Newspaper }\OtherTok{\textless{}{-}} \FunctionTok{lm}\NormalTok{(}\FunctionTok{unlist}\NormalTok{(dataframe[}\DecValTok{5}\NormalTok{]) }\SpecialCharTok{\textasciitilde{}} \FunctionTok{unlist}\NormalTok{(dataframe[}\DecValTok{4}\NormalTok{]) }\SpecialCharTok{*} \FunctionTok{unlist}\NormalTok{(dataframe[}\DecValTok{2}\NormalTok{]), }\AttributeTok{data =}\NormalTok{ dataframe)}
\FunctionTok{summary}\NormalTok{(lm\_model\_TV\_Newspaper)}
\end{Highlighting}
\end{Shaded}

\begin{verbatim}
## 
## Call:
## lm(formula = unlist(dataframe[5]) ~ unlist(dataframe[4]) * unlist(dataframe[2]), 
##     data = dataframe)
## 
## Residuals:
##     Min      1Q  Median      3Q     Max 
## -9.1860 -1.5521 -0.0648  1.8062  8.7276 
## 
## Coefficients:
##                                            Estimate Std. Error t value Pr(>|t|)
## (Intercept)                               6.4042175  0.7333818   8.732  1.1e-15
## unlist(dataframe[4])                      0.0241103  0.0192716   1.251    0.212
## unlist(dataframe[2])                      0.0426585  0.0043105   9.896  < 2e-16
## unlist(dataframe[4]):unlist(dataframe[2]) 0.0001324  0.0001079   1.228    0.221
##                                              
## (Intercept)                               ***
## unlist(dataframe[4])                         
## unlist(dataframe[2])                      ***
## unlist(dataframe[4]):unlist(dataframe[2])    
## ---
## Signif. codes:  0 '***' 0.001 '**' 0.01 '*' 0.05 '.' 0.1 ' ' 1
## 
## Residual standard error: 3.117 on 196 degrees of freedom
## Multiple R-squared:  0.6485, Adjusted R-squared:  0.6432 
## F-statistic: 120.6 on 3 and 196 DF,  p-value: < 2.2e-16
\end{verbatim}

\begin{Shaded}
\begin{Highlighting}[]
\CommentTok{\# R{-}squared = 0.6458 (or) 64.58\%; F{-}statistic = 120.6}

\NormalTok{lm\_model\_Radio\_Newspaper }\OtherTok{\textless{}{-}} \FunctionTok{lm}\NormalTok{(}\FunctionTok{unlist}\NormalTok{(dataframe[}\DecValTok{5}\NormalTok{]) }\SpecialCharTok{\textasciitilde{}} \FunctionTok{unlist}\NormalTok{(dataframe[}\DecValTok{4}\NormalTok{]) }\SpecialCharTok{*} \FunctionTok{unlist}\NormalTok{(dataframe[}\DecValTok{3}\NormalTok{]), }\AttributeTok{data =}\NormalTok{ dataframe)}
\FunctionTok{summary}\NormalTok{(lm\_model\_Radio\_Newspaper)}
\end{Highlighting}
\end{Shaded}

\begin{verbatim}
## 
## Call:
## lm(formula = unlist(dataframe[5]) ~ unlist(dataframe[4]) * unlist(dataframe[3]), 
##     data = dataframe)
## 
## Residuals:
##      Min       1Q   Median       3Q      Max 
## -15.6981  -2.1955   0.7567   2.7191   8.2228 
## 
## Coefficients:
##                                             Estimate Std. Error t value
## (Intercept)                                8.7904734  1.0224848   8.597
## unlist(dataframe[4])                       0.0220611  0.0345866   0.638
## unlist(dataframe[3])                       0.2145684  0.0382985   5.603
## unlist(dataframe[4]):unlist(dataframe[3]) -0.0005259  0.0010642  -0.494
##                                           Pr(>|t|)    
## (Intercept)                               2.58e-15 ***
## unlist(dataframe[4])                         0.524    
## unlist(dataframe[3])                      7.08e-08 ***
## unlist(dataframe[4]):unlist(dataframe[3])    0.622    
## ---
## Signif. codes:  0 '***' 0.001 '**' 0.01 '*' 0.05 '.' 0.1 ' ' 1
## 
## Residual standard error: 4.292 on 196 degrees of freedom
## Multiple R-squared:  0.3335, Adjusted R-squared:  0.3233 
## F-statistic:  32.7 on 3 and 196 DF,  p-value: < 2.2e-16
\end{verbatim}

\begin{Shaded}
\begin{Highlighting}[]
\CommentTok{\# R{-}squared = 0.3335 (or) 33.35\%; F{-}statistic = 32.7}

\DocumentationTok{\#\#\#\#\#\#\#\#\#\#\#\#\#\#\#\#\#\#\#\#\#\#\#\#\#\#\#\#\#\#\#\#\# }

\CommentTok{\# Question 5: Optimize sales}

\CommentTok{\# How should the budget be divided between TV \& Radio? }
\NormalTok{budget\_TV\_Radio }\OtherTok{\textless{}{-}} \FunctionTok{lm}\NormalTok{(}\FunctionTok{unlist}\NormalTok{(dataframe[}\DecValTok{5}\NormalTok{]) }\SpecialCharTok{\textasciitilde{}} \FunctionTok{unlist}\NormalTok{(dataframe[}\DecValTok{3}\NormalTok{]) }\SpecialCharTok{*} \FunctionTok{unlist}\NormalTok{(dataframe[}\DecValTok{2}\NormalTok{]), }\AttributeTok{data =}\NormalTok{ dataframe)}
\FunctionTok{summary}\NormalTok{(budget\_TV\_Radio)}
\end{Highlighting}
\end{Shaded}

\begin{verbatim}
## 
## Call:
## lm(formula = unlist(dataframe[5]) ~ unlist(dataframe[3]) * unlist(dataframe[2]), 
##     data = dataframe)
## 
## Residuals:
##     Min      1Q  Median      3Q     Max 
## -6.3366 -0.4028  0.1831  0.5948  1.5246 
## 
## Coefficients:
##                                            Estimate Std. Error t value Pr(>|t|)
## (Intercept)                               6.750e+00  2.479e-01  27.233   <2e-16
## unlist(dataframe[3])                      2.886e-02  8.905e-03   3.241   0.0014
## unlist(dataframe[2])                      1.910e-02  1.504e-03  12.699   <2e-16
## unlist(dataframe[3]):unlist(dataframe[2]) 1.086e-03  5.242e-05  20.727   <2e-16
##                                              
## (Intercept)                               ***
## unlist(dataframe[3])                      ** 
## unlist(dataframe[2])                      ***
## unlist(dataframe[3]):unlist(dataframe[2]) ***
## ---
## Signif. codes:  0 '***' 0.001 '**' 0.01 '*' 0.05 '.' 0.1 ' ' 1
## 
## Residual standard error: 0.9435 on 196 degrees of freedom
## Multiple R-squared:  0.9678, Adjusted R-squared:  0.9673 
## F-statistic:  1963 on 3 and 196 DF,  p-value: < 2.2e-16
\end{verbatim}

\begin{Shaded}
\begin{Highlighting}[]
\DocumentationTok{\#\# *I\textquotesingle{}m not sure how to answer this question* \#\#}


\DocumentationTok{\#\#\#\#\#\#\#\#\#\#\#\#\#\#\#\#\#\#\#\#\#\#\#\#\#\#\#\#\#\#\#\#\#}

\CommentTok{\# Part 2: Reading}

\DocumentationTok{\#\#\#\#\#\#\#\#\#\#\#\#\#\#\#\#\#\#\#\#\#\#\#\#\#\#\#\#\#\#\#\#\#}

\CommentTok{\# What is the goal of Machine Learning? }
\CommentTok{\# To develop high performance models that give useful predictions under computing restraints }

\CommentTok{\# What does Varian mean by "good out of sample predictions"?}
\CommentTok{\# It means to get good estimates or predictions on data that the model hasn\textquotesingle{}t seen yet. Sample here is the data with which }
\CommentTok{\# the model was estimated. So out of sample would mean data points outside this sample. }

\CommentTok{\# What is overfitting?}
\CommentTok{\# How Varian explains this is when a model fits linear indepedent variables perfectly with the training data, but don\textquotesingle{}t predict}
\CommentTok{\# well with data outside the training set, then the model is considered to be overfitting the training set. }

\CommentTok{\# What is model complexity?}
\CommentTok{\# If we visualize a model and observe one that has overfit, it will have a lot of depressions and curves so it touches all the }
\CommentTok{\# points. However, one that is not overfit or underfit, will look less twisted and bent with a fit which can be considered a good}
\CommentTok{\# one. So these are different comoplexities in models. }

\CommentTok{\# What is the training data?}
\CommentTok{\# The training data is the dataset with which we estimate our model}
\end{Highlighting}
\end{Shaded}


\end{document}
